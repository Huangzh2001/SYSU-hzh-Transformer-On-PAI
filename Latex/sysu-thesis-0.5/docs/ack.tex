%%
% 致谢
% 谢辞应以简短的文字对课题研究与论文撰写过程中曾直接给予帮助的人员(例如指导教师、答疑教师及其他人员)表示对自己的谢意,这不仅是一种礼貌,也是对他人劳动的尊重,是治学者应当遵循的学术规范。内容限一页。
% modifier: 黄俊杰
% update date: 2017-04-15
%%

\chapter{致谢}

首先我要感谢论文的指导老师时聪教授。在课题立项初期,我对当今的学术热点了解不深,时聪老师以其对整个研究领域的深刻理解和敏锐的洞察力给了我针对性的指导,帮助我完成了课题的选择及研究方向的确定。在项目实践的过程中,由于匮乏独自完成课题研究所需的经验,我常常请教时聪老师,在解决一项项学术问题的过程中,我不断提高着分析和解决问题的能力与养成了一定的科学素养,这使我在项目研究后期能开始独立完成课题的研究工作。这段经历的收获不在于学习到了多少具体的专业知识,而在于培养了我的科研思维和坚定了我选择学术研究道路的决心。

其次我还要感谢我本科的老师们,在此我郑重附上他们的名字:叶小平教授,刘长剑教授,陈秀卿教授,易泰山教授,赵育林教授,李铎教授,刘海峰教授,邵国宽教授,杨燕教授,魏国栋教授。他们不仅教会了我相关的专业知识,还培养了我数学的思维方式。

最后,我还要感谢我的家人,如果没有他们在背后的付出和支持,我就无法取得现在的成果。

\vskip 108pt
\begin{flushright}
	黄梓航\makebox[1cm]{} \\
	\today
\end{flushright}


\chapter{光声成像的理论基础}

\label{cha:sysu-thesis-contents-format-requirement}

\section{光声成像的波动方程}
假设$p(x,t)$为位于整个采集表面S上位置$x$的点在时刻$t$($t$≥0)的压力值。并且记位于表面S上位置$y$的点状声波探测器在观测时刻$t$获得的声波压力数据为函数$g(y,t)$,即$g(y,t):=p(y,t) \ for \ y\in S,t\geqslant0$。

于是我们得到一个波动方程:
\begin{equation} \label{201}
	\left\{
	\begin{aligned}
		& p_{tt}=c^2(x)\vartriangle _xp,\quad t\geqq0,x\in\mathbb{R}^3\\
		& p(x,0)=f(x),p_t(x,0)=0 \\
		& p|_s = g(y,t),\quad (y,t)\in S\times\mathbb{R}^+.
	\end{aligned}
	\right.
\end{equation}
其中,$f(x)$是声压的初始值。因此,光声成像的目标是使用传感器的测量数据$g(y,t)$,反推出上述波动方程$p(x,t)$在t=0处的初始值$f(x)$。

我们做如下记号:

定义1:我们使用$\mathcal{W}$表示正算子,即$\mathcal{W}:f(x)\to g(y,t)$,其中$f$与$g$的定义同上述波动方程。

如果成像介质是均匀的,即$c(x)$等于一个常数,我们假设该常数在适当的单位下等于1。此时,波动方程为:
\begin{equation} \label{202}
	\left\{
	\begin{aligned}
		& p_{tt}=\vartriangle _xp \quad \quad \ \ t\geqq0,x\in\mathbb{R}^3\\
		& p(x,0)=f(x),p_t(x,0)=0\\
		& p|_s = g(y,t),\quad (y,t)\in S\times\mathbb{R}^+.
	\end{aligned}
	\right.
\end{equation}


此时,根据$Poisson\ Kirchhoff$公式,我们能得到上述波动方程的解为:
\begin{equation} \label{203}
\begin{aligned}
p(x,t)=a\cfrac{\partial}{\partial t}(t(Rf)(x,t)).
\end{aligned}
\end{equation}
其中$(Rf)(x,r):=\cfrac{1}{4\pi}\int_{|y|=1} f(x+ry)\, dA(y)$是作用于函数$f(x)$的球面平均算子,$dA$是$\mathbb{R}^3$单位球面上的标准面积元,$a$为常数。

从上述公式可以得知,$p(x,t)$由函数$f$的球面平均值$(Rf)(x,t)$唯一决定。我们将这个这个球面平均算子作用在$f$上的映射$R:f→R$f记为$\mathcal{M}$,即:
\begin{equation} \label{204}
	\begin{aligned}
		\mathcal{M}f(x,t):=\cfrac{1}{4\pi}\int_{|y|=1} f(x+ry)\, dA(y),\quad x\in S,t\geqq0.
	\end{aligned}
\end{equation}
因此,在成像介质是均匀的情况下,我们可以选择使用$\mathcal{M}$来代替$p(x,t)$进行研究。

\section{光声成像的重建算法}
对于成像介质是均匀介质的情况(此时$c(x)$为常数),有上面的讨论可得,光声成像的图像重建等效于求解球面均值变换$\mathcal{M}$的逆。
下面介绍几种常见的光声成像重建方法:

\subsection{幂级数解法}
将$f$和$g$进行傅里叶分解后,即

\begin{equation} \label{205}
	\begin{aligned}
		f(x)=\sum_{-\infty}^{+\infty} f_k(\rho )e^{ik\varphi},\quad x=(\rho cos(\varphi),\rho sin(\varphi)).
	\end{aligned}
\end{equation}

\begin{equation} \label{206}
	\begin{aligned}
		g(y(\theta),r)=\sum_{-\infty}^{+\infty}g_k(r)e^{ik\theta},\quad y=(Rcos(\theta),Rsin(\theta)).
	\end{aligned}
\end{equation}

将其代入到公式(\ref{203})中,由等式两边系数相等可得:

\begin{equation} \label{207}
	\begin{aligned}
		f_k(\rho)=\mathcal{H}_m(\cfrac{1}{J_k(\lambda|R|)}\mathcal{H}_0\big[ \cfrac{g_k(r)}{2\pi r}\big]).
	\end{aligned}
\end{equation}

其中$(\mathcal{H}_mu)(s)=2\pi \int_{0}^{\infty}u(t)J_m(st)t dt$为Hankel变换,$J_m(t)$为贝塞尔函数。

应该注意的是,幂级数解法依赖于在球面几何中成立的变量分离,因此这种方法仅在球面上成立。


\subsection{特征函数展开法}

设$\lambda_m$和$u_m(x)$为封闭曲面S内部Ω的狄利克雷-拉普拉斯算子$−\vartriangle$的特征值和特征函数的正交基,满足:

\begin{equation} \label{208}
\left\{
\begin{aligned}
	& \vartriangle u_m(x)+\lambda_m^2u_m(x)=0,\quad x\in \Omega,\Omega\subseteq \mathbb{R}^n\\
	& u_m(x)=0,\quad \quad \quad \qquad \qquad x\in S\\
	& ||u_m||_2^2\equiv\int_{\Omega}|u_m(x)|^2dx=1.
\end{aligned}
\right.
\end{equation}

可解得:

\begin{equation} \label{209}
	\begin{aligned}
		u_m(x)=\int_{S}\Phi_{\lambda_m}(|x-y|)\cfrac{\partial}{\partial n}u_m(y)ds(y),\quad x\in\Omega.
	\end{aligned}
\end{equation}

其中$\Phi_{\lambda_m}(|x-y|)$是亥姆霍兹方程的自由空间格林函数,$n$是$S$的外法向量。

函数$f(x)$可以展开成级数:

\begin{equation} \label{210}
	\begin{aligned}
		f(x)=\sum_{m=0}^{\infty}\alpha_mu_m(x),\quad where\ \alpha_m=\int_{\Omega}u_m(x)f(x)dx.
	\end{aligned}
\end{equation}

如果将表示形式 ( \ref{209} ) 替换为表示形式( \ref{210} )并交换积分顺序,则可以得到$\alpha_m$。

\begin{equation} \label{211}
	\begin{aligned}
		& \alpha_m=\int_{\Omega}u_m(x)f(x)dx=\int_S I(y,\lambda_m)\cfrac{\partial}{\partial n}u_m(y)dA(x),\\
		& where\\
		& I(y,\lambda)=\int_{\Omega}\Phi_{\lambda}(|x-y|)f(x)dx=\int_0^{diam\Omega}g(y,r)\Phi_{\lambda}(r)dr.
	\end{aligned}
\end{equation}

将$\alpha_m$代入级数(\ref{210})就能得到重建公式$f(x)$。
